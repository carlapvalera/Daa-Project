\documentclass{article}
\usepackage[utf8]{inputenc}
\usepackage{amsmath}
\usepackage{verbatim}

\title{Solucionando el Problema de la Suma de Subconjuntos}
\author{Carla Sunami Pérez Valera}
\date{Septiembre 2024}

\begin{document}

\maketitle

\section{Introducción}
El Problema de la Suma de Subconjuntos es un problema clásico en la informática, que consiste en determinar si un subconjunto de números de un conjunto dado puede sumar un valor objetivo específico. Este problema es conocido por su complejidad computacional y es un ejemplo típico de un problema NP-completo, lo que significa que no se conoce una solución en tiempo polinómico para él.

\section{Enfoques Algorítmicos}
Hay varios enfoques para abordar el Problema de la Suma de Subconjuntos, cada uno con sus propias complejidades de tiempo y espacio:

\subsection{Enfoque de Backtracking }
El backtracking  es una forma metódica de probar varias secuencias de decisiones hasta encontrar una que "funcione". Al resolver el Problema de la Suma de Subconjuntos utilizando backtracking , exploramos todos los subconjuntos posibles de forma recursiva. Incluimos el elemento actual en el subconjunto y recurrimos para los elementos restantes con la suma restante. Si la suma se convierte en cero, imprimimos el subconjunto actual. También excluimos el elemento actual del subconjunto y recurrimos para los elementos restantes. La recursión se detiene cuando no quedan elementos o la suma se vuelve negativa.

Aquí hay un pseudocódigo similar a Python para el enfoque de retroceso:

\begin{verbatim}
def print_subset_sum(set, subset, n, target_sum):
    if target_sum == 0:
        print(subset)
        return
    if n == 0:
        return
    print_subset_sum(set, subset, n - 1, target_sum)
    if set[n - 1] <= target_sum:
        subset.append(set[n - 1])
        print_subset_sum(set, subset, n - 1, target_sum - set[n - 1])
        subset.pop()
\end{verbatim}

\subsection{Enfoque de Programación Dinámica}
La Programación Dinámica (DP) es una técnica de optimización que resuelve problemas descomponiéndolos en subproblemas más simples. Almacena los resultados de estos subproblemas para evitar cálculos redundantes. Para el Problema de la Suma de Subconjuntos, podemos usar un arreglo 2D de DP donde el estado \( dp[i][j] \) es verdadero si hay un subconjunto de elementos de \( set[0...i] \) con suma igual a \( j \).

Aquí hay un pseudocódigo similar a Python para el enfoque de DP:

\begin{verbatim}
def is_subset_sum(set, n, sum):
    dp = [[False for _ in range(sum + 1)] for _ in range(n + 1)]
    for i in range(n + 1):
        dp[i][0] = True
    for i in range(1, n + 1):
        for j in range(1, sum + 1):
            if j < set[i - 1]:
                dp[i][j] = dp[i - 1][j]
            else:
                dp[i][j] = dp[i - 1][j] or dp[i - 1][j - set[i - 1]]
    return dp[n][sum]
\end{verbatim}

\subsection{Optimización de Espacio en Programación Dinámica}
Para optimizar el espacio en el enfoque de DP, podemos usar un arreglo 1D en lugar de un arreglo 2D. Solo necesitamos la fila anterior para calcular los valores de la fila actual, por lo que podemos alternar entre dos arreglos para representar las filas anterior y actual.

Aquí hay un pseudocódigo similar a Python para el enfoque de DP optimizado en espacio:

\begin{verbatim}
def is_subset_sum(set, n, sum):
    prev = [False for _ in range(sum + 1)]
    prev[0] = True
    for i in range(1, n + 1):
        curr = [False for _ in range(sum + 1)]
        for j in range(1, sum + 1):
            if j < set[i - 1]:
                curr[j] = prev[j]
            else:
                curr[j] = prev[j] or prev[j - set[i - 1]]
        prev = curr
    return prev[sum]
\end{verbatim}

\section{Análisis de Complejidad}
La complejidad de tiempo del enfoque de backtracking  es \( O(2^n) \), ya que puede intentar todos los subconjuntos del conjunto dado en el peor de los casos. La complejidad de espacio es \( O(n) \) debido a la pila de recursión.

La complejidad de tiempo del enfoque de programación dinámica es \( O(sum * n) \), donde \( sum \) es la suma objetivo y \( n \) es el tamaño del arreglo. La complejidad de espacio es \( O(sum * n) \) para el arreglo 2D de DP y \( O(sum) \) para la versión optimizada en espacio.



\section{Conclusión}
El Problema de la Suma de Subconjuntos es un problema fundamental que muestra los desafíos del diseño y optimización de algoritmos. Aunque es computacionalmente intensivo, los diversos enfoques para resolverlo, desde el retroceso hasta la programación dinámica, proporcionan valiosas ideas sobre el pensamiento algorítmico y las estrategias de resolución de problemas.

\end{document}