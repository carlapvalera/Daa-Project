\documentclass{article}
\usepackage[utf8]{inputenc}
\usepackage{amsmath}
\usepackage{verbatim}

\title{Propuesta de Problemas de DAA}
\author{Carla Sunami Pérez Valera}
\date{Septiembre 2024}

\begin{document}

\maketitle

\section{Primer Problema}

\begin{center}
    \large \textbf{Dificultad:} 1900\\
    \LARGE \textbf{B1. Painting the Array I} \\[0.5em] 
    \large \textbf{time limit per test:} 2 seconds\\
    \large \textbf{memory limit per test:} 512 MB
\end{center}

\textbf{The only difference between the two versions is that this version asks the maximal possible answer.}

Homer likes arrays a lot. Today he is painting an array \( a_1, a_2, \ldots, a_n \) with two kinds of colors, white and black. A painting assignment for \( a_1, a_2, \ldots, a_n \) is described by an array \( b_1, b_2, \ldots, b_n \) that \( b_i \) indicates the color of \( a_i \) (0 for white and 1 for black).

According to a painting assignment \( b_1, b_2, \ldots, b_n \), the array \( a \) is split into two new arrays \( a(0) \) and \( a(1) \), where \( a(0) \) is the sub-sequence of all white elements in \( a \) and \( a(1) \) is the sub-sequence of all black elements in \( a \). For example, if \( a=[1,2,3,4,5,6] \) and \( b=[0,1,0,1,0,0] \), then \( a(0)=[1,3,5,6] \) and \( a(1)=[2,4] \).

The number of segments in an array \( c_1, c_2, \ldots, c_k \), denoted \( \text{seg}(c) \), is the number of elements if we merge all adjacent elements with the same value in \( c \). For example, the number of segments in \([1,1,2,2,3,3,3,2]\) is 4, because the array will become \([1,2,3,2]\) after merging adjacent elements with the same value. Especially, the number of segments in an empty array is 0.

Homer wants to find a painting assignment \( b \), according to which the number of segments in both \( a(0) \) and \( a(1) \), i.e. \( \text{seg}(a(0))+\text{seg}(a(1)) \), is as large as possible. Find this number.

\large \textbf{Input}\\
The first line contains an integer \( n \) \( (1 \leq n \leq 10^5) \).

La segunda línea contiene \( n \) enteros \( a_1, a_2, \ldots, a_n \) \( (1 \leq a_i \leq n) \).

\large \textbf{Output}\\
Output a single integer, indicating the maximal possible total number of segments.

\large \textbf{Examples}\\
\textbf{Input:}
\begin{verbatim}
7
1 1 2 2 3 3 3
\end{verbatim}

\textbf{Output:}
\begin{verbatim}
6
\end{verbatim}

\textbf{Input:}
\begin{verbatim}
7
1 2 3 4 5 6 7
\end{verbatim}

\textbf{Output:}
\begin{verbatim}
7
\end{verbatim}

\textbf{Note:} \\
In the first example, we can choose \( a(0)=[1,2,3,3] \), \( a(1)=[1,2,3] \) and \( \text{seg}(a(0))=\text{seg}(a(1))=3 \). So the answer is \( 3+3=6 \).

In the second example, we can choose \( a(0)=[1,2,3,4,5,6,7] \) and \( a(1) \) is empty. We can see that \( \text{seg}(a(0))=7 \) y \( \text{seg}(a(1))=0 \). So the answer is \( 7+0=7 \).

\section{Segundo Problema}

\begin{center}
    \LARGE \textbf{Problema de Monitoreo de Velocidad} \\[0.5em] 
\end{center}

No es broma cuando decimos que los accidentes automovilísticos se han vuelto cada vez más frecuentes en nuestro día a día. La continua violación de las leyes de tránsito o las carreras ilegales son solo algunas de las causas de estos incidentes. 

Ahora bien, se conoce que en una calle de longitud \( n \) metros, los conductores tienen la manía de sobrepasar el límite de velocidad señalado. Puesto que esta calle es una zona delicada, ya que es muy usada por niños, ancianos y el público en general, se quiere tener total control de que esto no ocurra. 

Para velar por el cumplimiento estricto del límite de velocidad, la alcaldía se ha dado a la tarea de instalar a lo largo de la calle cámaras de vigilancia con sensores de velocidad integrados, de modo tal que ni un metro de la calle quede sin ser monitoreado, para atrapar a cada conductor que infrinja la regla antes mencionada. Pero, obviamente, la alcaldía desea usar la mínima cantidad de recursos para llevar a cabo esta tarea y, quién sabe, tal vez ampliar este proyecto urbano a otra calle con el mismo problema.

Las cámaras que se instalarán tienen un rango de visibilidad (el cual es el mismo en ambos sentidos de la calle). Es decir, una cámara, una vez puesta en su poste, es capaz de supervisar la calle \( m \) metros hacia ambos lados. Por ejemplo, si ponemos una cámara a 7 metros del inicio de la calle y el rango de monitoreo es de 3 metros, entonces esa cámara monitoreará desde el 4º metro hasta el 10º metro de la calle (contando desde el inicio).
\textbf{Nota:} Las cámaras cerrán ubicadas en postes ya existentes para minimizar así el gasto de recursos.

\section{Tercer Problema de DAA}
\textbf{NPC problem}
\begin{center}
    \LARGE \textbf{Subset Sum Problem} \\[0.5em] 
\end{center}

Dado un arreglo de enteros no negativos y un valor \texttt{sum}, determina si hay un subconjunto del conjunto dado cuya suma sea igual al valor dado.

\textbf{Ejemplos:}

\textbf{Input:} \( n = 6, \text{ arr}[] = \{3, 34, 4, 12, 5, 2\}, \text{ sum} = 9 \) \\
\textbf{Output:} 1 \\
\textbf{Explicación:} Aquí existe un subconjunto con suma = 9, \( 4 + 3 + 2 = 9 \).

\vspace{10pt}

\textbf{Input:} \( n = 6, \text{ arr}[] = \{3, 34, 4, 12, 5, 2\}, \text{ sum} = 30 \) \\
\textbf{Output:} 0 \\
\textbf{Explicación:} No hay ningún subconjunto con suma 30.

\end{document}